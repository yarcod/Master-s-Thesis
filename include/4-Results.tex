\chapter{Results}
Simulating real applications in virtual hardware without any real interaction is hard and there are many things that can break along the way. The simulations were run on a by Chalmers provided compute cluster and took quite some time to complete. This cluster luckily used an older version of Ubuntu with a relatively old kernel, and since PIN 2 depends on old kernel functionality it meant it could be run without a hitch. In addition, in order to support the old ABI used with some of its binaries while having full support for C++11, GCC 4.9.4 had to be used, specifically. 
\bigskip

In this chapter we will first measure what latency behaviour looks like when using one single HMC device, with a single link hop. Then we will increase the number of devices and always use the device at the end of the chain, e.g. when having 8 devices we will perform all allocations on the 8th device. We will view these results both from a device and a application perspective. Finally, we present a short summary of our findings.

\section{Simulation}
TODO: How to present this? Segmentation?

\section{Summary}
Summarise the results without actually discussing the implications; leave that for the next part!