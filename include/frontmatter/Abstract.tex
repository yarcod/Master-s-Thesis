% CREATED BY DAVID FRISK, 2014
\THETITLE\\
\TITLEDESCR\\
DANIEL EDHOLM\\
Department of Computer Science and Engineering\\
Chalmers University of Technology\\

\thispagestyle{plain}			% Supress header 
\section*{Abstract}
With emerging memory technologies aiming to bring high bandwidth memory in close proximity of the processor, regular access latency is altered and application performance and/or behaviours could change. One approach to this near memory computation is using an abstract, packet based protocol which both allows the memory controller to be placed at the memory and for packets to be routed. Using a network of memories would allow a main memory to be scalable and be able to add new hosts anywhere in the network, but on the other hand distances, and thus latencies, are affected by this integration method. In this thesis we have set up a simulated network of stacked DRAM memories based on the HMC protocol and have measured how latencies are affected by link contention as well as scale with added nodes. Our results indicate that without a proper governer of memory accesses, Memory Level Parallelism can hurt performance. In addition, this penalty his hidden when having to perform multiple hops. Furthermore, contention is created even when using a memory device close to the host and increase somewhat with additional hops in the network. 

% KEYWORDS (MAXIMUM 10 WORDS)
\vfill
Keywords: High bandwidth memory, Hybrid memory cube, network, latency, contention, stacked memories.

\newpage				% Create empty back of side
\thispagestyle{empty}
\mbox{}