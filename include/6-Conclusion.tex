\chapter{Conclusion}
In this work we have evaluated how applications behave, and to some extent perform, when run on a network of memories. Such a network will inherently add latency and thus slow the execution time down; especially in this case since we measured a worst-case scenario using Hybrid Memory Cube devices in a chain topology. Our approach has also included non-optmised software and naive scheduling, and thereby visualising issues arising from longer memory traversal time as much as possible. Our simulated results suggests that trying to utilise Memory Level Parallelism (MLP) without intelligent scheduling or awareness can worsen performance by increasing average latency. The abstract, packet based protocol used enables the CPU simulator to send memory requests through any link and still expect a valid response, since the memory cubes themselves contain internal routing capabilities to the proper vault. Without awareness, the overhead created with routing can decrease application performance. 
\bigskip

Although a high-speed interface is used, contention is created and decreases performance relative to the memory intensity of an application. Furthermore, contention increases somewhat with added network hops, but the largest impact is seen when having to jump more than once from the host. The observed behaviour where more links worsen performance is masked when having to traverse the network further, and as such the MLP bottleneck of having no scheduling is cancelled at this point. However, there are still no improvements with contention using more links, which further justifies the need for intelligent scheduling. When requests are in-flight in the network there need to be some sort of awareness of preferred links inside the bottom logic die in all nodes. 
\bigskip

In the end, the biggest increase to latency comes neither from contention nor underutilisation of MLP, but from each added hop in the network. Specifically, this is only the latency in the memory network and not how big the latency impact is on application run time. In addition, this is while running a single application and the results are expected to change once multiple programs run simultaneously, creating even more contention.